% Options for packages loaded elsewhere
\PassOptionsToPackage{unicode}{hyperref}
\PassOptionsToPackage{hyphens}{url}
%
\documentclass[
]{article}
\usepackage{amsmath,amssymb}
\usepackage{iftex}
\ifPDFTeX
  \usepackage[T1]{fontenc}
  \usepackage[utf8]{inputenc}
  \usepackage{textcomp} % provide euro and other symbols
\else % if luatex or xetex
  \usepackage{unicode-math} % this also loads fontspec
  \defaultfontfeatures{Scale=MatchLowercase}
  \defaultfontfeatures[\rmfamily]{Ligatures=TeX,Scale=1}
\fi
\usepackage{lmodern}
\ifPDFTeX\else
  % xetex/luatex font selection
\fi
% Use upquote if available, for straight quotes in verbatim environments
\IfFileExists{upquote.sty}{\usepackage{upquote}}{}
\IfFileExists{microtype.sty}{% use microtype if available
  \usepackage[]{microtype}
  \UseMicrotypeSet[protrusion]{basicmath} % disable protrusion for tt fonts
}{}
\makeatletter
\@ifundefined{KOMAClassName}{% if non-KOMA class
  \IfFileExists{parskip.sty}{%
    \usepackage{parskip}
  }{% else
    \setlength{\parindent}{0pt}
    \setlength{\parskip}{6pt plus 2pt minus 1pt}}
}{% if KOMA class
  \KOMAoptions{parskip=half}}
\makeatother
\usepackage{xcolor}
\usepackage[margin=1in]{geometry}
\usepackage{color}
\usepackage{fancyvrb}
\newcommand{\VerbBar}{|}
\newcommand{\VERB}{\Verb[commandchars=\\\{\}]}
\DefineVerbatimEnvironment{Highlighting}{Verbatim}{commandchars=\\\{\}}
% Add ',fontsize=\small' for more characters per line
\usepackage{framed}
\definecolor{shadecolor}{RGB}{248,248,248}
\newenvironment{Shaded}{\begin{snugshade}}{\end{snugshade}}
\newcommand{\AlertTok}[1]{\textcolor[rgb]{0.94,0.16,0.16}{#1}}
\newcommand{\AnnotationTok}[1]{\textcolor[rgb]{0.56,0.35,0.01}{\textbf{\textit{#1}}}}
\newcommand{\AttributeTok}[1]{\textcolor[rgb]{0.13,0.29,0.53}{#1}}
\newcommand{\BaseNTok}[1]{\textcolor[rgb]{0.00,0.00,0.81}{#1}}
\newcommand{\BuiltInTok}[1]{#1}
\newcommand{\CharTok}[1]{\textcolor[rgb]{0.31,0.60,0.02}{#1}}
\newcommand{\CommentTok}[1]{\textcolor[rgb]{0.56,0.35,0.01}{\textit{#1}}}
\newcommand{\CommentVarTok}[1]{\textcolor[rgb]{0.56,0.35,0.01}{\textbf{\textit{#1}}}}
\newcommand{\ConstantTok}[1]{\textcolor[rgb]{0.56,0.35,0.01}{#1}}
\newcommand{\ControlFlowTok}[1]{\textcolor[rgb]{0.13,0.29,0.53}{\textbf{#1}}}
\newcommand{\DataTypeTok}[1]{\textcolor[rgb]{0.13,0.29,0.53}{#1}}
\newcommand{\DecValTok}[1]{\textcolor[rgb]{0.00,0.00,0.81}{#1}}
\newcommand{\DocumentationTok}[1]{\textcolor[rgb]{0.56,0.35,0.01}{\textbf{\textit{#1}}}}
\newcommand{\ErrorTok}[1]{\textcolor[rgb]{0.64,0.00,0.00}{\textbf{#1}}}
\newcommand{\ExtensionTok}[1]{#1}
\newcommand{\FloatTok}[1]{\textcolor[rgb]{0.00,0.00,0.81}{#1}}
\newcommand{\FunctionTok}[1]{\textcolor[rgb]{0.13,0.29,0.53}{\textbf{#1}}}
\newcommand{\ImportTok}[1]{#1}
\newcommand{\InformationTok}[1]{\textcolor[rgb]{0.56,0.35,0.01}{\textbf{\textit{#1}}}}
\newcommand{\KeywordTok}[1]{\textcolor[rgb]{0.13,0.29,0.53}{\textbf{#1}}}
\newcommand{\NormalTok}[1]{#1}
\newcommand{\OperatorTok}[1]{\textcolor[rgb]{0.81,0.36,0.00}{\textbf{#1}}}
\newcommand{\OtherTok}[1]{\textcolor[rgb]{0.56,0.35,0.01}{#1}}
\newcommand{\PreprocessorTok}[1]{\textcolor[rgb]{0.56,0.35,0.01}{\textit{#1}}}
\newcommand{\RegionMarkerTok}[1]{#1}
\newcommand{\SpecialCharTok}[1]{\textcolor[rgb]{0.81,0.36,0.00}{\textbf{#1}}}
\newcommand{\SpecialStringTok}[1]{\textcolor[rgb]{0.31,0.60,0.02}{#1}}
\newcommand{\StringTok}[1]{\textcolor[rgb]{0.31,0.60,0.02}{#1}}
\newcommand{\VariableTok}[1]{\textcolor[rgb]{0.00,0.00,0.00}{#1}}
\newcommand{\VerbatimStringTok}[1]{\textcolor[rgb]{0.31,0.60,0.02}{#1}}
\newcommand{\WarningTok}[1]{\textcolor[rgb]{0.56,0.35,0.01}{\textbf{\textit{#1}}}}
\usepackage{graphicx}
\makeatletter
\def\maxwidth{\ifdim\Gin@nat@width>\linewidth\linewidth\else\Gin@nat@width\fi}
\def\maxheight{\ifdim\Gin@nat@height>\textheight\textheight\else\Gin@nat@height\fi}
\makeatother
% Scale images if necessary, so that they will not overflow the page
% margins by default, and it is still possible to overwrite the defaults
% using explicit options in \includegraphics[width, height, ...]{}
\setkeys{Gin}{width=\maxwidth,height=\maxheight,keepaspectratio}
% Set default figure placement to htbp
\makeatletter
\def\fps@figure{htbp}
\makeatother
\setlength{\emergencystretch}{3em} % prevent overfull lines
\providecommand{\tightlist}{%
  \setlength{\itemsep}{0pt}\setlength{\parskip}{0pt}}
\setcounter{secnumdepth}{-\maxdimen} % remove section numbering
\usepackage{tcolorbox}
\usepackage{listings}
\lstset{ breaklines=true, breakatwhitespace=true, keepspaces=true, xleftmargin=0pt, xrightmargin=0pt, frame=single, columns=fullflexible, linewidth=\dimexpr\textwidth+1in\relax }
\usepackage[none]{hyphenat}
\usepackage[dvipsnames]{xcolor}
\sloppy
\AtBeginDocument{\geometry{a4paper, top=1in, bottom=1.5in, left=0.75in, right=0.75in}}
\AtBeginDocument{\setlength{\textwidth}{6.5in}}
\linespread{1.2}
\usepackage[hidelinks]{hyperref} % Assure des liens actifs mais discrets
\hypersetup{colorlinks=true, linkcolor=blue, urlcolor=blue, filecolor=blue, citecolor=blue}
\usepackage{amsmath}
\usepackage{amssymb}
\usepackage{graphicx}
\usepackage{fontspec}
\setmainfont{Cambria}
\renewcommand{\normalsize}{\fontsize{12}{14}\selectfont}
\setsansfont{Franklin Gothic Demi Cond}
\setmonofont{Courier New}
\usepackage{titlesec}
\titleformat{\section}{\Huge\bfseries\color{blue}}{\thesection}{1em}{}
\titleformat{\subsection}{\huge\bfseries\color{blue}}{\thesubsection}{1em}{}
\titleformat{\subsubsection}{\LARGE\bfseries\color{blue}}{\thesubsubsection}{1em}{}
\usepackage{tocloft}
\renewcommand{\cftsecfont}{\LARGE}
\renewcommand{\cftsubsecfont}{\Large}
\renewcommand{\cftsecfont}{\small}
\renewcommand{\cftsubsecfont}{\footnotesize}
\renewcommand{\cftsecpagefont}{\small}
\renewcommand{\cftsubsecpagefont}{\footnotesize}
\renewcommand{\cftsecleader}{\cftdotfill{\cftdotsep}}
\ifLuaTeX
  \usepackage{selnolig}  % disable illegal ligatures
\fi
\usepackage{bookmark}
\IfFileExists{xurl.sty}{\usepackage{xurl}}{} % add URL line breaks if available
\urlstyle{same}
\hypersetup{
  hidelinks,
  pdfcreator={LaTeX via pandoc}}

\author{}
\date{\vspace{-2.5em}}

\begin{document}

\begin{titlepage}
    \begin{center}
        \sffamily 
        {\Large \textbf{RÉPUBLIQUE DU SÉNÉGAL}}\\[0.3cm]
        \includegraphics[width=3cm]{LOGO3.jpg} \\[0.3cm]
        
        {\large \textbf{Un Peuple - Un But - Une Foi}}\\[0.5cm]
        
        {\Large \textbf{Ministère de l'Économie, du Plan et de la Coopération}}\\[0.5cm]
        
        \includegraphics[width=3cm]{LOGO2.jpg} \\[0.3cm]
        
        {\Large \textbf{Agence Nationale de la Statistique et de la Démographie (ANSD)}}\\[0.5cm]
        
        \includegraphics[width=3cm]{LOGO1.jpg} \\[0.3cm]
        
        {\LARGE \textbf{École Nationale de la Statistique et de l'Analyse Économique (ENSAE | Pierre Ndiaye)}}\\[0.8cm]
        
        \textbf{\Huge \color{blue} TP4}\\[0.6cm]
        \rule{\linewidth}{0.6mm} \\[1cm]
        
        \vfill  
        
        \begin{minipage}{0.5\textwidth}
    \begin{flushleft} \large
        \emph{\textsf{Rédigé par :}}\\
        \textbf{FOGWOUNG DJOUFACK Sarah-Laure}\\
        \textbf{NIASS Ahmadou}\\
        \textbf{NGUEMFOUO NGOUMTSA Célina}\\
        \textbf{SENE Malick}\\
        \textit{Élèves Ingénieurs Statisticiens Économistes}
    \end{flushleft}
\end{minipage}
        \hfill
        \begin{minipage}{0.4\textwidth}
            \begin{flushright} \large
                \emph{\textsf{Sous la supervision de :}} \\
                \textbf{M. Aboubacar HEMA}\\
                \textit{Research Analyst }
            \end{flushright}
        \end{minipage}

        \vfill

        {\large \textsf{Année scolaire : 2024/2025}}\\[0.5cm]
        
    \end{center}
\end{titlepage}

\section{TABLE DE MATIERES}\label{table-de-matieres}

\tableofcontents

\newpage

\section{CONSIGNE DU TP4}\label{consigne-du-tp4}

Ce TP consiste à faire correspondre les codes \textbf{ADM3\_PCODE} de
chaque commune des pays \textbf{Burkina Faso} et \textbf{Niger} entre
deux bases de données :

\begin{itemize}
\tightlist
\item
  \textbf{Base EHCVM} : Contient des informations détaillées sur les
  conditions de vie, les revenus, et d'autres aspects socio-économiques
  des ménages.
\item
  \textbf{Base HDX} : Contient des données administratives avec les
  codes \textbf{ADM3\_PCODE}, souvent utilisées pour le suivi
  humanitaire et les informations géographiques et administratives des
  communes.
\end{itemize}

Le \textbf{problème principal} est que les noms des communes ne sont pas
toujours écrits de la même façon dans les deux bases. Pour résoudre
cela, il faut regarder aux niveaux département et région afin de trouver
les correspondances exactes.

L'objectif à atteindre est donc de \textbf{fusionner les deux bases de
données en s'assurant que les codes ADM3\_PCODE de chaque commune soient
bien alignés}.

\newpage

\section{STEP 1: IMPORTATION DES
PACKAGES}\label{step-1-importation-des-packages}

Dans cette étape, nous importons plusieurs packages qui nous permettront
d'effectuer le travail demandé. Voici un aperçu de ce que chaque package
apporte :

\begin{itemize}
\tightlist
\item
  \textbf{Le package haven} : est particulièrement utile pour importer
  des fichiers \texttt{.dta} (Stata) en R.
\item
  \textbf{Le package readxl} : permet de lire les fichiers Excel
  (formats .xls et .xlsx).
\item
  \textbf{Le package dplyr} :permet de filtrer, sélectionner, regrouper,
  trier, et effectuer d'autres opérations de transformation, facilitant
  ainsi la manipulation des données.
\item
  \textbf{Le package stringr} : fournit des fonction pour la
  manipublation des chaines de caractères (traitement de textes,
  recherche/substitution/modification des chaines de caractères \ldots)
\item
  \textbf{Le package sf} : fournit des outils pour lire, manipuler et
  analyser des données spatiales
\item
  \textbf{Le package tidyverse} : collection de packages qui facilitent
  la manipulation, la visualisation et l'analyse des données. Il inclut
  des packages comme ggplot2 (pour la visualisation), dplyr (pour la
  manipulation des données), tidyr (pour le nettoyage des données), et
  plus encore.
\item
  \textbf{Le package stringi} : une version plus complète du package
  stringr pour la manipulation des chaînes de caractères. Il propose une
  plus grande variété de fonctions pour le traitement de texte et est
  conçu pour être plus performant dans des cas plus complexes.
\end{itemize}

\begin{Shaded}
\begin{Highlighting}[]
\FunctionTok{library}\NormalTok{(haven)}
\FunctionTok{library}\NormalTok{(readxl)}
\FunctionTok{library}\NormalTok{(dplyr)}
\FunctionTok{library}\NormalTok{(stringr)}
\FunctionTok{library}\NormalTok{(sf)}
\FunctionTok{library}\NormalTok{(tidyverse) }
\FunctionTok{library}\NormalTok{(stringi)}
\end{Highlighting}
\end{Shaded}

\newpage

\section{STEP 2: TRAITEMENT POUR LE
NIGER}\label{step-2-traitement-pour-le-niger}

\subsection{2-a CHARGEMENT DES BASES DE
DONNÉES}\label{a-chargement-des-bases-de-donnuxe9es}

Nous commençons par charger les bases de données en utilisant les
packages appropriés. La première base est au format Stata (utilisée pour
les données de l'EHCVM), et la deuxième est au format Excel (utilisée
pour les données de la base HDX).

\begin{Shaded}
\begin{Highlighting}[]
\NormalTok{ehcvm\_raw }\OtherTok{\textless{}{-}}\NormalTok{ haven}\SpecialCharTok{::}\FunctionTok{read\_dta}\NormalTok{(}
  \StringTok{"BASE DE DONNEES/s00\_me\_ner2021.dta"}\NormalTok{) }
\NormalTok{hdx\_raw }\OtherTok{\textless{}{-}}\NormalTok{ readxl}\SpecialCharTok{::}\FunctionTok{read\_excel}\NormalTok{(}
  \StringTok{"BASE DE DONNEES/ner\_admgz\_ignn\_20230720.xlsx"}\NormalTok{) }
\end{Highlighting}
\end{Shaded}

\subsection{2-b TRAITEMENT DES COMMUNES DANS LA BASE
EHCVM}\label{b-traitement-des-communes-dans-la-base-ehcvm}

Nous nettoyons les noms des communes dans la base EHCVM en appliquant
plusieurs transformations : - Conversion des codes des communes en
labels. - Mise en minuscules, suppression des accents, des apostrophes
et des caractères non alphanumériques. - Suppression des espaces en
début et en fin, ainsi que des espaces multiples. - Correction des noms
de certaines communes.

\begin{Shaded}
\begin{Highlighting}[]
\NormalTok{ehcvm }\OtherTok{\textless{}{-}}\NormalTok{ ehcvm\_raw }\SpecialCharTok{\%\textgreater{}\%}
\NormalTok{  dplyr}\SpecialCharTok{::}\FunctionTok{mutate}\NormalTok{(}
    \AttributeTok{commune\_label =} \FunctionTok{as\_factor}\NormalTok{(s00q03),}
    \AttributeTok{commune\_clean =}\NormalTok{ commune\_label }\SpecialCharTok{|\textgreater{}}
\NormalTok{      stringr}\SpecialCharTok{::}\FunctionTok{str\_to\_lower}\NormalTok{() }\SpecialCharTok{|\textgreater{}}
\NormalTok{      (\textbackslash{}(x) }\FunctionTok{iconv}\NormalTok{(x, }\AttributeTok{to =} \StringTok{"ASCII//TRANSLIT"}\NormalTok{))() }\SpecialCharTok{|\textgreater{}}
\NormalTok{      stringr}\SpecialCharTok{::}\FunctionTok{str\_replace\_all}\NormalTok{(}\StringTok{"\textquotesingle{}"}\NormalTok{, }\StringTok{""}\NormalTok{) }\SpecialCharTok{|\textgreater{}}
\NormalTok{      stringr}\SpecialCharTok{::}\FunctionTok{str\_replace\_all}\NormalTok{(}\StringTok{"[\^{}[:alnum:] ]"}\NormalTok{, }\StringTok{" "}\NormalTok{) }\SpecialCharTok{|\textgreater{}}
\NormalTok{      stringr}\SpecialCharTok{::}\FunctionTok{str\_trim}\NormalTok{() }\SpecialCharTok{|\textgreater{}}
\NormalTok{      stringr}\SpecialCharTok{::}\FunctionTok{str\_replace\_all}\NormalTok{(}\StringTok{"}\SpecialCharTok{\textbackslash{}\textbackslash{}}\StringTok{barrondissement}\SpecialCharTok{\textbackslash{}\textbackslash{}}\StringTok{b"}\NormalTok{, }\StringTok{""}\NormalTok{) }\SpecialCharTok{|\textgreater{}}
\NormalTok{      stringr}\SpecialCharTok{::}\FunctionTok{str\_replace\_all}\NormalTok{(}\StringTok{"}\SpecialCharTok{\textbackslash{}\textbackslash{}}\StringTok{b1}\SpecialCharTok{\textbackslash{}\textbackslash{}}\StringTok{b"}\NormalTok{, }\StringTok{"i"}\NormalTok{) }\SpecialCharTok{|\textgreater{}}
\NormalTok{      stringr}\SpecialCharTok{::}\FunctionTok{str\_replace\_all}\NormalTok{(}\StringTok{"}\SpecialCharTok{\textbackslash{}\textbackslash{}}\StringTok{b2}\SpecialCharTok{\textbackslash{}\textbackslash{}}\StringTok{b"}\NormalTok{, }\StringTok{"ii"}\NormalTok{) }\SpecialCharTok{|\textgreater{}}
\NormalTok{      stringr}\SpecialCharTok{::}\FunctionTok{str\_replace\_all}\NormalTok{(}\StringTok{"}\SpecialCharTok{\textbackslash{}\textbackslash{}}\StringTok{b3}\SpecialCharTok{\textbackslash{}\textbackslash{}}\StringTok{b"}\NormalTok{, }\StringTok{"iii"}\NormalTok{) }\SpecialCharTok{|\textgreater{}}
\NormalTok{      stringr}\SpecialCharTok{::}\FunctionTok{str\_replace\_all}\NormalTok{(}\StringTok{"}\SpecialCharTok{\textbackslash{}\textbackslash{}}\StringTok{b4}\SpecialCharTok{\textbackslash{}\textbackslash{}}\StringTok{b"}\NormalTok{, }\StringTok{"iv"}\NormalTok{) }\SpecialCharTok{|\textgreater{}}
\NormalTok{      stringr}\SpecialCharTok{::}\FunctionTok{str\_replace\_all}\NormalTok{(}\StringTok{"}\SpecialCharTok{\textbackslash{}\textbackslash{}}\StringTok{b5}\SpecialCharTok{\textbackslash{}\textbackslash{}}\StringTok{b"}\NormalTok{, }\StringTok{"v"}\NormalTok{) }\SpecialCharTok{|\textgreater{}}
\NormalTok{      stringr}\SpecialCharTok{::}\FunctionTok{str\_squish}\NormalTok{()}
\NormalTok{  )}
\end{Highlighting}
\end{Shaded}

\subsection{2-c CORRECTION ET TRAITEMENT DES COMMUNES DANS LA BASE
HDX}\label{c-correction-et-traitement-des-communes-dans-la-base-hdx}

De même, dans la base HDX, nous corrigeons certains noms de communes en
fonction des départements et régions correspondants. Une série de règles
est appliquée pour certains cas spécifiques. Nous nettoyons également
les noms de communes dans la base HDX de la même manière que pour EHCVM.

\begin{Shaded}
\begin{Highlighting}[]
\NormalTok{hdx }\OtherTok{\textless{}{-}}\NormalTok{ hdx\_raw }\SpecialCharTok{\%\textgreater{}\%}
\NormalTok{  dplyr}\SpecialCharTok{::}\FunctionTok{mutate}\NormalTok{(}
    \AttributeTok{ADM3\_FR =}\NormalTok{ dplyr}\SpecialCharTok{::}\FunctionTok{case\_when}\NormalTok{(}
\NormalTok{      ADM3\_FR }\SpecialCharTok{==} 
        \StringTok{"Tibiri"} \SpecialCharTok{\&}\NormalTok{ ADM1\_FR }\SpecialCharTok{==} \StringTok{"Dosso"} \SpecialCharTok{\textasciitilde{}} \StringTok{"Tibiri Doutchi"}\NormalTok{,}
\NormalTok{      ADM3\_FR }\SpecialCharTok{==} 
        \StringTok{"Tibiri"} \SpecialCharTok{\&}\NormalTok{ ADM1\_FR }\SpecialCharTok{==} \StringTok{"Maradi"} \SpecialCharTok{\textasciitilde{}} \StringTok{"Tibiri Maradi"}\NormalTok{,}
\NormalTok{      ADM3\_FR }\SpecialCharTok{==} 
        \StringTok{"Gangara"} \SpecialCharTok{\&}\NormalTok{ ADM1\_FR }\SpecialCharTok{==} \StringTok{"Maradi"} \SpecialCharTok{\textasciitilde{}} \StringTok{"Gangara Gazaoua"}\NormalTok{,}
\NormalTok{      ADM3\_FR }\SpecialCharTok{==} 
        \StringTok{"Gangara"} \SpecialCharTok{\&}\NormalTok{ ADM1\_FR }\SpecialCharTok{==} \StringTok{"Zinder"} \SpecialCharTok{\textasciitilde{}} \StringTok{"Gangara Tanout"}\NormalTok{,}
      \ConstantTok{TRUE} \SpecialCharTok{\textasciitilde{}}\NormalTok{ ADM3\_FR}
\NormalTok{    ),}
    
    \CommentTok{\# Création de la colonne corrigée commune\_clean}
    \AttributeTok{commune\_clean =}\NormalTok{ ADM3\_FR }\SpecialCharTok{|\textgreater{}}
\NormalTok{      stringr}\SpecialCharTok{::}\FunctionTok{str\_to\_lower}\NormalTok{() }\SpecialCharTok{|\textgreater{}}
\NormalTok{      (\textbackslash{}(x) }\FunctionTok{iconv}\NormalTok{(x, }\AttributeTok{to =} \StringTok{"ASCII//TRANSLIT"}\NormalTok{))() }\SpecialCharTok{|\textgreater{}}
\NormalTok{      stringr}\SpecialCharTok{::}\FunctionTok{str\_replace\_all}\NormalTok{(}\StringTok{"\textquotesingle{}"}\NormalTok{, }\StringTok{""}\NormalTok{) }\SpecialCharTok{|\textgreater{}}
\NormalTok{      stringr}\SpecialCharTok{::}\FunctionTok{str\_replace\_all}\NormalTok{(}\StringTok{"[\^{}[:alnum:] ]"}\NormalTok{, }\StringTok{" "}\NormalTok{) }\SpecialCharTok{|\textgreater{}}
\NormalTok{      stringr}\SpecialCharTok{::}\FunctionTok{str\_trim}\NormalTok{() }\SpecialCharTok{|\textgreater{}}
\NormalTok{      stringr}\SpecialCharTok{::}\FunctionTok{str\_squish}\NormalTok{()}
\NormalTok{  )}
\end{Highlighting}
\end{Shaded}

\subsection{2-d JOINTURE DES BASES}\label{d-jointure-des-bases}

Nous effectuons une jointure des deux bases (EHCVM et HDX) en utilisant
la colonne ``commune\_clean'' comme clé de correspondance. Ici, nous
utilisons une jointure gauche (left\_join) pour conserver toutes les
communes de la base EHCVM.

\begin{Shaded}
\begin{Highlighting}[]
\NormalTok{ehcvm\_hdx\_merged }\OtherTok{\textless{}{-}}\NormalTok{ dplyr}\SpecialCharTok{::}\FunctionTok{left\_join}\NormalTok{(}
\NormalTok{  ehcvm, hdx, }\AttributeTok{by =} \StringTok{"commune\_clean"}\NormalTok{)}
\end{Highlighting}
\end{Shaded}

\subsection{2-e VERIFICATION DES COMMUNES NON
APPARIEES}\label{e-verification-des-communes-non-appariees}

Une fois la jointure effectuée, nous vérifions les communes qui ne se
sont pas appariées. Nous filtrons et affichons les lignes où
``ADM3\_FR'' est NA, ce qui signifie qu'il n'y a pas eu de
correspondance.Et on verra qu'il y a eu des correspondances partout.

\begin{Shaded}
\begin{Highlighting}[]
\NormalTok{non\_matchees }\OtherTok{\textless{}{-}}\NormalTok{ dplyr}\SpecialCharTok{::}\FunctionTok{filter}\NormalTok{(}
\NormalTok{  ehcvm\_hdx\_merged, }\FunctionTok{is.na}\NormalTok{(ADM3\_FR)) }\SpecialCharTok{\%\textgreater{}\%} 
\NormalTok{  dplyr}\SpecialCharTok{::}\FunctionTok{distinct}\NormalTok{(s00q03, commune\_clean) }

\NormalTok{non\_matchees}
\end{Highlighting}
\end{Shaded}

\begin{verbatim}
## # A tibble: 0 x 2
## # i 2 variables: s00q03 <dbl+lbl>, commune_clean <chr>
\end{verbatim}

On constate que toutes les communes sont matchées.

\newpage

\section{STEP 3: TRAITEMENT POUR LE BURKINA
FASO}\label{step-3-traitement-pour-le-burkina-faso}

\subsection{3-a CHARGEMENT DES BASES DE
DONNÉES}\label{a-chargement-des-bases-de-donnuxe9es-1}

Nous commençons par charger les bases de données en utilisant les
packages appropriés. Comme précedement, la première base est au format
Stata (utilisée pour les données de l'EHCVM), et la deuxième est au
format Excel (utilisée pour les données de la base HDX).

\begin{Shaded}
\begin{Highlighting}[]
\NormalTok{ehcvm\_raw }\OtherTok{\textless{}{-}}\NormalTok{ haven}\SpecialCharTok{::}\FunctionTok{read\_dta}\NormalTok{(}
  \StringTok{"BASE DE DONNEES/s00\_me\_bfa2021.dta"}\NormalTok{)}
\NormalTok{shape\_raw }\OtherTok{\textless{}{-}}\NormalTok{ readxl}\SpecialCharTok{::}\FunctionTok{read\_excel}\NormalTok{(}
  \StringTok{"BASE DE DONNEES/bfa\_adminboundaries\_tabulardata.xlsx"}\NormalTok{, }
  \AttributeTok{sheet =} \StringTok{"ADM3"}\NormalTok{)}
\end{Highlighting}
\end{Shaded}

\subsection{3-b CORRECTION DES ERREURS
D'ORTHOGRAPHE}\label{b-correction-des-erreurs-dorthographe}

L'étape suivant est la correction des erreurs des erreurs d'orthographe
dans la base EHCVM.

Après avoir analysé les noms des communes, nous avons identifié
certaines incohérences orthographiques mineures, comme l'utilisation
d'un ``y'' à la place d'un ``i'' ou d'autres variations de
transcription. Afin d'harmoniser ces noms, nous avons créé une table de
correspondance permettant de corriger ces différences.

Par ailleurs, pour les arrondissements numérotés de 1 à 12, nos
recherches ont montré qu'ils appartiennent tous à la commune de
Ouagadougou. Nous avons donc choisi de les uniformiser en remplaçant
leur nom par ``Ouagadougou'' dans notre base de données. Pour ce faire
la fonction \emph{correction\_map} a été crée.

\begin{itemize}
\tightlist
\item
  \textbf{Création de la fonction correction\_map}
\end{itemize}

\begin{Shaded}
\begin{Highlighting}[]
\NormalTok{correction\_map }\OtherTok{\textless{}{-}} \FunctionTok{data.frame}\NormalTok{(}
  \AttributeTok{commune\_label =} \FunctionTok{c}\NormalTok{(}
    \StringTok{"Zeguedeguin"}\NormalTok{,}\StringTok{"Bondigui"}\NormalTok{,}\StringTok{"Absouya"}\NormalTok{, }\StringTok{"Bondokuy"}\NormalTok{, }
    \StringTok{"Bomborokuy"}\NormalTok{, }\StringTok{"Bittou"}\NormalTok{, }\StringTok{"Bokin"}\NormalTok{, }\StringTok{"Boudry"}\NormalTok{,}
    \StringTok{"Bobo Dioulasso{-}Konsa"}\NormalTok{, }\StringTok{"Bobo Dioulasso{-}Dô"}\NormalTok{, }\StringTok{"Arbole"}\NormalTok{, }
    \StringTok{"Dapelogo"}\NormalTok{, }\StringTok{"Dissin"}\NormalTok{, }\StringTok{"Fada N\textquotesingle{}gourma"}\NormalTok{, }\StringTok{"Gounghin"}\NormalTok{,}
    \StringTok{"Kokoloko"}\NormalTok{, }\StringTok{"Gourcy"}\NormalTok{, }\StringTok{"La{-}Todin"}\NormalTok{, }\StringTok{"Karankasso{-}Vigue"}\NormalTok{,}
    \StringTok{"Sanga"}\NormalTok{, }\StringTok{"Meguet"}\NormalTok{, }\StringTok{"Soaw"}\NormalTok{, }\StringTok{"Samorogouan"}\NormalTok{, }\StringTok{"Sabce"}\NormalTok{,}
    \StringTok{"Tanghin Dassouri"}\NormalTok{, }\StringTok{"Oury"}\NormalTok{, }\StringTok{"Cassou"}\NormalTok{,}
    \StringTok{"Arrondissement 1"}\NormalTok{, }\StringTok{"Arrondissement 2"}\NormalTok{,}
    \StringTok{"Arrondissement 3"}\NormalTok{, }\StringTok{"Arrondissement 4"}\NormalTok{,}
    \StringTok{"Arrondissement 5"}\NormalTok{, }\StringTok{"Arrondissement 6"}\NormalTok{,}
    \StringTok{"Arrondissement 7"}\NormalTok{, }\StringTok{"Arrondissement 8"}\NormalTok{,}
    \StringTok{"Arrondissement 9"}\NormalTok{, }\StringTok{"Arrondissement 10"}\NormalTok{,}
    \StringTok{"Arrondissement 11"}\NormalTok{, }\StringTok{"Arrondissement 12"}\NormalTok{,}
    \StringTok{"Arrondissement N 1"}\NormalTok{, }\StringTok{"Arrondissement N 2"}\NormalTok{,}
    \StringTok{"Arrondissement N 3"}\NormalTok{, }\StringTok{"Arrondissement N 4"}\NormalTok{,}
    \StringTok{"Arrondissement N 5"}\NormalTok{, }\StringTok{"Arrondissement N 6"}\NormalTok{,}
    \StringTok{"Arrondissement N 7"}\NormalTok{),}
  \AttributeTok{corrected\_name =} \FunctionTok{c}\NormalTok{(}
    \StringTok{"Senguènèga"}\NormalTok{,}\StringTok{"Bondokui"}\NormalTok{,}\StringTok{"Ambsouya"}\NormalTok{, }\StringTok{"Bondokui"}\NormalTok{, }
    \StringTok{"Bomborokui"}\NormalTok{, }\StringTok{"Bitou"}\NormalTok{, }\StringTok{"Boken"}\NormalTok{, }\StringTok{"Boudri"}\NormalTok{,}
    \StringTok{"Bobo{-}Dioulasso"}\NormalTok{, }\StringTok{"Bobo{-}Dioulasso"}\NormalTok{, }\StringTok{"Arbollé"}\NormalTok{,}
    \StringTok{"Dapeolgo"}\NormalTok{, }\StringTok{"Dissihn"}\NormalTok{, }\StringTok{"Fada Ngourma"}\NormalTok{, }\StringTok{"Gounguen"}\NormalTok{,}
    \StringTok{"Kokologo"}\NormalTok{, }\StringTok{"Goursi"}\NormalTok{, }\StringTok{"La{-}Toden"}\NormalTok{, }\StringTok{"Karangasso{-}Vigué"}\NormalTok{,}
    \StringTok{"Sangha"}\NormalTok{, }\StringTok{"Mégué"}\NormalTok{, }\StringTok{"Soa"}\NormalTok{, }\StringTok{"Samôgôgouan"}\NormalTok{, }\StringTok{"Sabsé"}\NormalTok{,}
    \StringTok{"Tanguen{-}Dassouri"}\NormalTok{, }\StringTok{"Ouri"}\NormalTok{, }\StringTok{"Kassou"}\NormalTok{, }\StringTok{"Ouagadougou"}\NormalTok{, }
    \StringTok{"Ouagadougou"}\NormalTok{, }\StringTok{"Ouagadougou"}\NormalTok{, }\StringTok{"Ouagadougou"}\NormalTok{,}
    \StringTok{"Ouagadougou"}\NormalTok{, }\StringTok{"Ouagadougou"}\NormalTok{, }\StringTok{"Ouagadougou"}\NormalTok{,}
    \StringTok{"Ouagadougou"}\NormalTok{, }\StringTok{"Ouagadougou"}\NormalTok{, }\StringTok{"Ouagadougou"}\NormalTok{,}
    \StringTok{"Ouagadougou"}\NormalTok{, }\StringTok{"Ouagadougou"}\NormalTok{, }\StringTok{"Ouagadougou"}\NormalTok{,}
    \StringTok{"Ouagadougou"}\NormalTok{, }\StringTok{"Ouagadougou"}\NormalTok{, }\StringTok{"Ouagadougou"}\NormalTok{,}
    \StringTok{"Ouagadougou"}\NormalTok{, }\StringTok{"Ouagadougou"}\NormalTok{, }\StringTok{"Ouagadougou"}\NormalTok{)}
\NormalTok{)}
\end{Highlighting}
\end{Shaded}

\begin{itemize}
\tightlist
\item
  \textbf{Application de la fonction à la base EHCVM}
\end{itemize}

\begin{Shaded}
\begin{Highlighting}[]
\NormalTok{ehcvm\_corrected }\OtherTok{\textless{}{-}}\NormalTok{ ehcvm\_raw }\SpecialCharTok{\%\textgreater{}\%}
\NormalTok{  dplyr}\SpecialCharTok{::}\FunctionTok{left\_join}\NormalTok{(correction\_map, }
                   \AttributeTok{by =} \FunctionTok{c}\NormalTok{(}\StringTok{"s00q03"} \OtherTok{=} \StringTok{"commune\_label"}\NormalTok{)) }\SpecialCharTok{\%\textgreater{}\%}
\NormalTok{  dplyr}\SpecialCharTok{::}\FunctionTok{mutate}\NormalTok{(}\AttributeTok{s00q03 =} \FunctionTok{ifelse}\NormalTok{(}\FunctionTok{is.na}\NormalTok{(corrected\_name),}
\NormalTok{                                s00q03, corrected\_name)) }\SpecialCharTok{\%\textgreater{}\%}
\NormalTok{  dplyr}\SpecialCharTok{::}\FunctionTok{select}\NormalTok{(}\SpecialCharTok{{-}}\NormalTok{corrected\_name)}
\end{Highlighting}
\end{Shaded}

\subsection{3-c NETTOYAGE DES DONNES}\label{c-nettoyage-des-donnes}

Nous nettoyons ensuite les noms des communes et départements des bases
EHCVM et HDX, notament en tenant compte de la casse, supprimant les
accents, caractères spéciaux, apostrophes et espaces inutiles. Pour ce
faire, la fonction \emph{clean\_names} a été crée.

\begin{itemize}
\tightlist
\item
  \textbf{Création de la fonction}
\end{itemize}

\begin{Shaded}
\begin{Highlighting}[]
\NormalTok{clean\_names }\OtherTok{\textless{}{-}} \ControlFlowTok{function}\NormalTok{(name) \{}
\NormalTok{  name }\SpecialCharTok{\%\textgreater{}\%}
\NormalTok{    stringr}\SpecialCharTok{::}\FunctionTok{str\_to\_lower}\NormalTok{() }\SpecialCharTok{\%\textgreater{}\%}
    \FunctionTok{iconv}\NormalTok{(}\AttributeTok{to =} \StringTok{"ASCII//TRANSLIT"}\NormalTok{) }\SpecialCharTok{\%\textgreater{}\%} 
\NormalTok{    stringr}\SpecialCharTok{::}\FunctionTok{str\_replace\_all}\NormalTok{(}\StringTok{"\textquotesingle{}"}\NormalTok{, }\StringTok{""}\NormalTok{) }\SpecialCharTok{\%\textgreater{}\%}
\NormalTok{    stringr}\SpecialCharTok{::}\FunctionTok{str\_replace\_all}\NormalTok{(}\StringTok{"[{-}]"}\NormalTok{, }\StringTok{" "}\NormalTok{) }\SpecialCharTok{\%\textgreater{}\%}
\NormalTok{    stringr}\SpecialCharTok{::}\FunctionTok{str\_replace\_all}\NormalTok{(}\StringTok{"[\^{}[:alnum:] ]"}\NormalTok{, }\StringTok{" "}\NormalTok{) }\SpecialCharTok{\%\textgreater{}\%}
\NormalTok{    stringr}\SpecialCharTok{::}\FunctionTok{str\_trim}\NormalTok{() }\SpecialCharTok{\%\textgreater{}\%}
\NormalTok{    stringr}\SpecialCharTok{::}\FunctionTok{str\_squish}\NormalTok{()}
\NormalTok{\}}
\end{Highlighting}
\end{Shaded}

\begin{itemize}
\tightlist
\item
  \textbf{Traitement de la base EHCVM}
\end{itemize}

\begin{Shaded}
\begin{Highlighting}[]
\CommentTok{\# EHCVM : convertion des codes en labels et nettoyage}
\NormalTok{ehcvm }\OtherTok{\textless{}{-}}\NormalTok{ ehcvm\_corrected }\SpecialCharTok{\%\textgreater{}\%}
\NormalTok{  dplyr}\SpecialCharTok{::}\FunctionTok{mutate}\NormalTok{(}
    \AttributeTok{commune\_label =}\NormalTok{ forcats}\SpecialCharTok{::}\FunctionTok{as\_factor}\NormalTok{(s00q03),}
    \AttributeTok{departement\_label =}\NormalTok{ forcats}\SpecialCharTok{::}\FunctionTok{as\_factor}\NormalTok{(s00q02),}
    \AttributeTok{commune\_clean =} \FunctionTok{clean\_names}\NormalTok{(commune\_label),}
    \AttributeTok{departement\_clean =} \FunctionTok{clean\_names}\NormalTok{(departement\_label)}
\NormalTok{  )}
\end{Highlighting}
\end{Shaded}

\begin{itemize}
\tightlist
\item
  \textbf{Traitement de la base HDX}
\end{itemize}

\begin{Shaded}
\begin{Highlighting}[]
\CommentTok{\# Shape Data : nettoyage des communes et départements}
\NormalTok{shape }\OtherTok{\textless{}{-}}\NormalTok{ shape\_raw }\SpecialCharTok{\%\textgreater{}\%}
\NormalTok{  dplyr}\SpecialCharTok{::}\FunctionTok{mutate}\NormalTok{(}
    \AttributeTok{commune\_clean =} \FunctionTok{clean\_names}\NormalTok{(ADM3\_FR),}
    \AttributeTok{departement\_clean =} \FunctionTok{clean\_names}\NormalTok{(ADM2\_FR)}
\NormalTok{  )}
\end{Highlighting}
\end{Shaded}

\subsection{3-d FUSION}\label{d-fusion}

Une fois toutes les données nettoyées, nous effectuons une jointure sur
la base EHCVM afin d'obtenir les noms corrects des communes dans la
base.

Toutefois, il existe des communes necessitant des vérifications par
département : il s'agit des communes boussouma, namissiguima et thiou.

\begin{Shaded}
\begin{Highlighting}[]
\NormalTok{communes\_ambigues }\OtherTok{\textless{}{-}} \FunctionTok{c}\NormalTok{(}\StringTok{"boussouma"}\NormalTok{, }\StringTok{"namissiguima"}\NormalTok{, }\StringTok{"thiou"}\NormalTok{)}
\end{Highlighting}
\end{Shaded}

Une fois ces communes identifiées, un premier merge sera effectué sur
les autres communes.

\begin{Shaded}
\begin{Highlighting}[]
\CommentTok{\# Premier merge}
\NormalTok{ehcvm\_shape\_merged }\OtherTok{\textless{}{-}}\NormalTok{ ehcvm }\SpecialCharTok{\%\textgreater{}\%}
\NormalTok{  dplyr}\SpecialCharTok{::}\FunctionTok{filter}\NormalTok{(}\SpecialCharTok{!}\NormalTok{commune\_clean }\SpecialCharTok{\%in\%}\NormalTok{ communes\_ambigues) }\SpecialCharTok{\%\textgreater{}\%}
\NormalTok{  dplyr}\SpecialCharTok{::}\FunctionTok{left\_join}\NormalTok{(shape, }\AttributeTok{by =} \StringTok{"commune\_clean"}\NormalTok{)}
\end{Highlighting}
\end{Shaded}

Une fusion est à présent faite sur les communes ambigües en tenant
compte des départements.

\begin{Shaded}
\begin{Highlighting}[]
\CommentTok{\# Fusion pour les communes ambigües}
\NormalTok{ehcvm\_shape\_merged\_ambigues }\OtherTok{\textless{}{-}}\NormalTok{ ehcvm }\SpecialCharTok{\%\textgreater{}\%}
\NormalTok{  dplyr}\SpecialCharTok{::}\FunctionTok{filter}\NormalTok{(commune\_clean }\SpecialCharTok{\%in\%}\NormalTok{ communes\_ambigues) }\SpecialCharTok{\%\textgreater{}\%}
\NormalTok{  dplyr}\SpecialCharTok{::}\FunctionTok{left\_join}\NormalTok{(shape, }\AttributeTok{by =} \FunctionTok{c}\NormalTok{(}
    \StringTok{"commune\_clean"}\NormalTok{, }\StringTok{"departement\_clean"}\NormalTok{))}

\CommentTok{\# Combiner les deux bases après fusion}
\NormalTok{ehcvm\_shape\_final }\OtherTok{\textless{}{-}}\NormalTok{ dplyr}\SpecialCharTok{::}\FunctionTok{bind\_rows}\NormalTok{(}
\NormalTok{  ehcvm\_shape\_merged, ehcvm\_shape\_merged\_ambigues)}
\end{Highlighting}
\end{Shaded}

Une fois la fusion terminée, finalisons le travail en faisant des
vérifications et en supprimant les colonnes inutiles.

\subsection{3-e VERFICATION}\label{e-verfication}

Vérifions si toutes les communes ont été vusionnées

\begin{Shaded}
\begin{Highlighting}[]
\NormalTok{non\_matchees }\OtherTok{\textless{}{-}}\NormalTok{ ehcvm\_shape\_final }\SpecialCharTok{\%\textgreater{}\%}
\NormalTok{  dplyr}\SpecialCharTok{::}\FunctionTok{filter}\NormalTok{(}\FunctionTok{is.na}\NormalTok{(ADM3\_PCODE)) }\SpecialCharTok{\%\textgreater{}\%}
\NormalTok{  dplyr}\SpecialCharTok{::}\FunctionTok{distinct}\NormalTok{(s00q03, departement\_label, commune\_clean)}

\FunctionTok{head}\NormalTok{(non\_matchees)}
\end{Highlighting}
\end{Shaded}

\begin{verbatim}
## # A tibble: 0 x 3
## # i 3 variables: s00q03 <chr>, departement_label <fct>, commune_clean <chr>
\end{verbatim}

Toutes les communes ont été matchées. La dernière étape, la suppression
des colonnes temporaires peut enfin être éffectuée.

\subsection{3-f SUPPRESSION DES COLONNES
TEMPORAIRES}\label{f-suppression-des-colonnes-temporaires}

Les colonnes temporaires à supprimer sont: communes\_clean,
departement\_clean et commune\_label.

\begin{Shaded}
\begin{Highlighting}[]
\NormalTok{ehcvm\_final }\OtherTok{\textless{}{-}}\NormalTok{ ehcvm\_shape\_final }\SpecialCharTok{\%\textgreater{}\%}
\NormalTok{  dplyr}\SpecialCharTok{::}\FunctionTok{select}\NormalTok{(}
    \SpecialCharTok{{-}}\NormalTok{commune\_clean, }\SpecialCharTok{{-}}\NormalTok{departement\_clean, }\SpecialCharTok{{-}}\NormalTok{commune\_label,}
    \SpecialCharTok{{-}}\NormalTok{departement\_label)}
\end{Highlighting}
\end{Shaded}

Voici un aperçu de la base finale:

\begin{Shaded}
\begin{Highlighting}[]
\FunctionTok{print}\NormalTok{(}\FunctionTok{head}\NormalTok{(ehcvm\_final))}
\end{Highlighting}
\end{Shaded}

\begin{verbatim}
## # A tibble: 6 x 45
##     hhid grappe menage vague hhweight s00q00     s00q01  s00q02   s00q03 s00q04 
##    <dbl>  <dbl>  <dbl> <dbl>    <dbl> <dbl+lbl>  <dbl+l> <dbl+lb> <chr>  <dbl+l>
## 1 586005    586      5     2      439 2 [Burkin~ 2 [Bou~ 13 [Kos~ Djiba~ 2 [Rur~
## 2 586028    586     28     2      439 2 [Burkin~ 2 [Bou~ 13 [Kos~ Djiba~ 2 [Rur~
## 3 586043    586     43     2      439 2 [Burkin~ 2 [Bou~ 13 [Kos~ Djiba~ 2 [Rur~
## 4 586044    586     44     2      439 2 [Burkin~ 2 [Bou~ 13 [Kos~ Djiba~ 2 [Rur~
## 5 586052    586     52     2      439 2 [Burkin~ 2 [Bou~ 13 [Kos~ Djiba~ 2 [Rur~
## 6 586082    586     82     2      439 2 [Burkin~ 2 [Bou~ 13 [Kos~ Djiba~ 2 [Rur~
## # i 35 more variables: s00q05 <chr>, s00q07a <dbl+lbl>, s00q07b <dbl+lbl>,
## #   s00q07c <dbl+lbl>, s00q07d <dbl+lbl>, s00q07d2 <dbl+lbl>, s00q22 <dbl>,
## #   s00q23a <chr>, s00q24a <chr>, s00q25a <chr>, s00q23b <chr>, s00q24b <chr>,
## #   s00q25b <chr>, s00q08 <dbl+lbl>, s00q27 <dbl+lbl>, s00q28 <dbl+lbl>,
## #   GPS__Latitude <dbl>, GPS__Longitude <dbl>, departement_clean.x <chr>,
## #   ADM3_FR <chr>, ADM3_PCODE <chr>, ADM3_REF <chr>, ADM3ALT1_FR <lgl>,
## #   ADM3ALT2_FR <lgl>, ADM2_FR <chr>, ADM2_PCODE <chr>, ADM1_FR <chr>, ...
\end{verbatim}

\end{document}
